\documentclass[a4paper, 11pt]{article} % Font size (can be 10pt, 11pt or 12pt) and paper size (remove a4paper for US letter paper)

\usepackage[protrusion=true,expansion=true]{microtype} % Better typography
\usepackage{graphicx} % Required for including pictures
\usepackage{wrapfig} % Allows in-line images
\usepackage{enumitem} %%Enables control over enumerate and itemize environments
\usepackage{setspace}
\usepackage{amssymb, amsmath, mathrsfs, mathabx} %%Math packages
\usepackage{stmaryrd}
\usepackage{mathtools}
\usepackage{mathpazo} % Use the Palatino font
\usepackage[T1]{fontenc} % Required for accented characters
\usepackage{array}
\usepackage{bibentry}
\usepackage[round]{natbib} %%Or change 'round' to 'square' for square backers
\setcitestyle{aysep={}}

\linespread{1.05} % Change line spacing here, Palatino benefits from a slight increase by default


\usepackage{xcolor}

\definecolor{codegreen}{rgb}{0,0.6,0}
\definecolor{codegray}{rgb}{0.5,0.5,0.5}
\definecolor{codepurple}{rgb}{0.58,0,0.82}
\definecolor{backcolour}{rgb}{0.95,0.95,0.92}

\usepackage{listings}
\lstdefinestyle{mystyle}{
    backgroundcolor=\color{backcolour},   
    commentstyle=\color{codegreen},
    keywordstyle=\color{magenta},
    numberstyle=\tiny\color{codegray},
    stringstyle=\color{codepurple},
    basicstyle=\ttfamily\footnotesize,
    breakatwhitespace=false,         
    breaklines=true,                 
    captionpos=b,                    
    keepspaces=true,                 
    % numbers=left,                    
    numbersep=5pt,                  
    showspaces=false,                
    showstringspaces=false,
    showtabs=false,                  
    tabsize=2
}

\lstset{style=mystyle}


\DeclareSymbolFont{symbolsC}{U}{txsyc}{m}{n}
\SetSymbolFont{symbolsC}{bold}{U}{txsyc}{bx}{n}
\DeclareFontSubstitution{U}{txsyc}{m}{n}
\DeclareMathSymbol{\boxright}{\mathrel}{symbolsC}{"80}
\DeclareMathSymbol{\circleright}{\mathrel}{symbolsC}{"91}
\DeclareMathSymbol{\diamondright}{\mathrel}{symbolsC}{"84}
\DeclareMathSymbol{\medcirc}{\mathrel}{symbolsC}{"07}

\newcommand{\corner}[1]{\ulcorner#1\urcorner} %%Corner quotes
\newcommand{\tuple}[1]{\langle#1\rangle} %%Angle brackets
% \newcommand{\brak}[1]{[#1]} %%Angle brackets
\newcommand{\set}[1]{\lbrace#1\rbrace} %%Set brackets
% \newcommand{\At}[1]{\texttt{At}(#1)} %%Set brackets
\newcommand{\interpret}[1]{\llbracket#1\rrbracket} %%Double brackets
\renewcommand{\P}[0]{\mathbb{P}}
\renewcommand{\max}[0]{\texttt{Max}}
\newcommand{\Lit}[0]{\mathbb{Lit}}
\newcommand{\arity}[1]{\texttt{Arity}(#1)}
\newcommand{\M}[0]{\mathcal{M}}
\renewcommand{\L}[0]{\mathcal{L}}
\newcommand{\Q}[0]{\mathcal{Q}}
\newcommand{\N}[0]{\mathbb{N}}
\renewcommand{\O}[0]{\mathbb{O}}
\renewcommand{\Vert}[1]{\ldbrack#1\rdbrack}
\renewcommand{\vert}[1]{\lvert#1\rvert}
%\DeclarePairedDelimiter\ceil{\lceil}{\rceil}    

\makeatletter
\renewcommand\@biblabel[1]{\textbf{#1.}} % Change the square brackets for each bibliography item from '[1]' to '1.'
\renewcommand{\@listI}{\itemsep=0pt} % Reduce the space between items in the itemize and enumerate environments and the bibliography

\renewcommand{\maketitle}{ % Customize the title - do not edit title and author name here, see the TITLE block below
\begin{flushright} % Right align
{\LARGE\@title} % Increase the font size of the title

\vspace{10pt} % Some vertical space between the title and author name

{\@author} % Author name
\\\@date % Date

\vspace{10pt} % Some vertical space between the author block and abstract
\end{flushright}
}

%----------------------------------------------------------------------------------------
%	TITLE
%----------------------------------------------------------------------------------------

\title{\textbf{Counterfactual Models}} % Subtitle

\author{\textsc{UROP:} Miguel Buitrago\\ \em Benjamin Brast-McKie} % Institution

\date{\today} % Date

%----------------------------------------------------------------------------------------

\begin{document}

\maketitle % Print the title section

\thispagestyle{empty}

%----------------------------------------------------------------------------------------

% TODO: reshape to focus on set evaluator
  % mention integration with Z3
  % mention integration with Leen

\noindent
\textbf{Overview:} 
This is a preliminary outline describing the scope of the project and providing a sense of what motivates the construction.
The primary aim of the project is to use Z3 to evaluate entailments between counterfactual claims up to some limit in model complexity, providing countermodels when they exist.
By ruling out countermodels up to a specified level of complexity, we may provide evidence that the entailment of interest is valid over all models.
A second aim is to provide a method for identifying all entailments up to a specified level of syntactic complexity that do not admit countermodels below a specified value for model complexity.
The entailments within the specified value of syntactic complexity may then serve as the basis upon which to identify a logic for the language.
A third and final aim is to adequately document the project, taking steps towards generalising the methodology so that a user could investigate the logic for a language by defining its models and semantic clauses. 

A preliminary outline for each part will be provided below and is subject to change.
Settling the details given below constitutes the first part of the project, though these details may continue to change throughout.






\section{Sentence Encoding}

% It will be important to fix a syntax for the propositional language $\L$ in question.
% To avoid reinventing the wheel, some investigation should be conducted to see if there are standard ways to encode propositional sentences.
The following definitions will use \textit{lists} to encode the propositional sentences for a language which includes $n$ distinct sentence letters.

\begin{enumerate}
  \item[\it Signature:] The language is specified by providing a list of operators $\Q=\tuple{Q_1,Q_2,\ldots}$ where each $Q_i$ has a finite $\arity{Q_i}$.
  \item[\it Language:] Letting $\Lit_n=\set{p_i:1\leq i\leq n}$ be the first $n$ \textit{sentence letters}, we may refer to $\L_n=\tuple{\Lit_n,\Q}$ as a language of \textit{rank} $n$ with operators $\Q$.
  \item[\it Grammar:] We may encode the sentences of $\L_n$ in infix notation as follows:
    \begin{itemize}
      \item If $A\in\Lit_n$, then $[A]$ is a sentence of $\L_n$.
      \item If $\arity{Q}=k$ and $B_i$ is a sentence of $\L_n$ for each $1\leq i\leq k$, then $[Q,B_1,\ldots,B_k]$ is a sentence of $\L_n$.
    \end{itemize}
  \item[\it Translation:] It would be convenient have an algorithm for translating sentences from standard notation into polish notation so that $( 1 \vee 2 ) \boxright 3$ can be transformed back and forth with $[\boxright, \tuple{\vee, \tuple{1}, \tuple{2}}, \tuple{3}]$.
  % \item[\it Complexity:] It will also be important to have an algorithm for computing the complexity of the sentences of $\L_n$
\end{enumerate}

\section{Bitvector Frames}

This section defines frames in terms of \textit{bitvectors}, i.e., sequences of $1$s and $0$s of length $n$, where a $1$ in the $i$\textsuperscript{th} value of a bitvector may be interpreted as the inclusion of an \textit{atomic specific proposition} which is a basic way for things things to be, and a $0$ in the $i$\textsuperscript{th} value indicates neutrality (not falsity).

\begin{enumerate}
  \item[\it States:] An $n$-\textit{state} $s=\tuple{s_1,\ldots,s_n}$ is any bitvector of $1$s and $0$s of length $n$.
  \item[\it Atomic States:] A state which includes exactly one occurrence of $1$ is \textit{atomic} where this corresponds to a specific way of being, e.g., the cat is on the mat.
    % The property \texttt{is_atomic} may be defined for states as follows:
  \begin{lstlisting}[language=Python]

      def is_atomic(bit_vector):
        return And(
            x != 0, 0 == (x & (x - 1))
        )
    \end{lstlisting}
  % \item[\it Null State:] The \textit{null $n$-state} $\square=\tuple{0,\ldots,0}$ obtains trivially since it includes nothing.
  \item[\it Parthood:] An $n$-state $s$ is \textit{part of} an $n$-state $t$ (i.e., $s\sqsubseteq t$) \textit{iff} $s_i\leq t_i$ for all $1\leq i\leq n$.
    \begin{itemize}
      \item An $n$-state $s$ is \textit{part of} an $n$-state $t$ \textit{iff} $(s \texttt{ OR } t) = t$  where \texttt{OR} is the bitwise operator where: $11101010 \texttt{ OR } 00000100 = 11101110$. %the bitwise operation \texttt{OR} may be used to define parthood , though : $11101010 \texttt{ AND } 11111101 = 11101000$.
    \end{itemize}
  \item[\it Fusion:] Letting $X$ be a set of $n$-states and $\max_i(X)=\max(\set{s_i:s\in X})$ be the maximum of the $i$\textsuperscript{th} entries, the \textit{fusion} $\bigsqcup X=\tuple{\max_1(X),\ldots,\max_n(X)}$ includes the maximum entry for each of the $n$-states in $X$.
    We may represent the fusion $\bigsqcup\set{s,t,\ldots}=s.t.\ldots$ for ease of exposition.
    \begin{itemize}
      \item The \textit{fusion} $s.t = (s \texttt{ OR } t)$ is also easily defined, where the fusion of $\bigsqcup X$ may be defined by successive applications of \texttt{OR}.
      \item Are their other bitwise operations that might be helpful?
      % \item The bitwise operations \texttt{AND} may also be useful, though it is not clear for what. % : $11101010 \texttt{ AND } 11111101 = 11101000$.
    \end{itemize}
  \item[\it State Space:] A set of $n$-states $S$ forms a \textit{state space iff} $\bigsqcup X\in S$ for all $X\subseteq S$. Trivially, the set of all parts of any $n$-state form a state space.
  \item[\it Possible:] $\tuple{S,P}$ is a \textit{modal $n$-frame iff} $S$ is a state space and $P\subseteq S$ is a nonempty set of \textit{possible states} where $s\in P$ whenever $s\sqsubseteq t$ for some $t\in P$.
  \item[\it Compatible:] $s$ and $t$ are \textit{compatible iff} their fusion is possible, i.e., $s\circ t\coloneq s.t\in P$.
  \item[\it World States:] $w$ is a \textit{world state} over the $n$-frame $\tuple{S,P}$ \textit{iff} $w$ is possible and every state $t$ that is compatible with $w$ is part of $w$. 
    Thus each modal $n$-frame determines a set of world states $W=\set{w\in P: s\sqsubseteq w \text{ whenever } s\circ w}$.
  \item[\it Exhaustive:] $V,F\subseteq S$ are \textit{exhaustive} over the $n$-frame $\tuple{S,P}$ \textit{iff} every $s\in P$ is compatible with a state $t$ in either $V$ or $F$.
  \item[\it Exclusive:] $V,F\subseteq S$ are \textit{exclusive} over the $n$-frame $\tuple{S,P}$ \textit{iff} the states in $V$ are incompatible with the states in $F$.
  \item[\it Closed:] $X$ is closed over a state space $S$ \textit{iff} $\bigsqcup Y\in X$ whenever $Y\subseteq X$.
  \item[\it Propositions:] $\tuple{V,F}$ is a \textit{proposition} over the $n$-frame $\tuple{S,P}$ \textit{iff} $V,F\subseteq S$ are closed, exhaustive, and exclusive.
    Let $\P_{\tuple{S,P}}$ be the propositions over $\tuple{S,P}$.
  % \item[\it Order:] Given an $n$-frame $\tuple{A_n,P}$, the cardinality of $P$ is the possibility $p$ of the frame. Thus for each value of $n$, we may generate and store a list of all $n$-frames ordered by their possibility. Each $n$-frame should be given an hash and stored together with the set of states $S$, set of world states $W$, and set of propositions $\P$ that it determines.
  % \item[\it Bypass:] As above, each $n$-frame should be stored with a bypass parameter that can be turned on if later deemed to be uninteresting.
  % \item[\it Summary:] Each atomic number $n$ will generate a list of $n$-frames ordered by possibility $p$, where these $n$-frames are to be stored together with the states $S$, worlds states $W$, and propositions $\P$ that they determine together with a hash and a bypass parameter. 
\end{enumerate}




\section{Counterfactual Semantics}

This section defines $n$-models in terms of $n$-frames, providing the semantics for a propositional language with a counterfactual conditional operator.

\begin{enumerate}
  \item[\it Model:] $\tuple{S,P,\vert{\cdot}}$ is a \textit{bitvector $n$-model} of the propositional language $\L_m$ \textit{iff} $\tuple{S,P}$ is an $n$-frame and $\vert{p}\in\P_{\tuple{S,P}}$ for every sentence letter $p\in\Lit_m$.
  \item[\it Propositional Operators:] We may then define the following operators for $X,Y,V,F\subseteq S$:
    \begin{itemize}
      \item[($\otimes$)] $X \otimes Y \coloneq \set{s.t : s \in X,\ t \in Y}$.
      \item[($\oplus$)] $X \oplus Y \coloneq X \cup Y \cup (X \otimes Y)$.
      \item[($\neg$)] $\neg\tuple{V,F} \coloneq \tuple{F,V}$.
      \item[($\wedge$)] $\tuple{V,F}\wedge\tuple{V',F'} \coloneq \tuple{V\otimes V',F\oplus F'}$.
      \item[($\vee$)] $\tuple{V,F}\vee\tuple{V',F'} \coloneq \tuple{V\oplus V',F\otimes F'}$.
    \end{itemize}
    % These definitions provide operations on the space of propositions.
  \item[\it Extensional Semantics:] Letting $\L_n=\tuple{\Lit_n,\neg,\wedge,\vee}$ be an extensional language, each $n$-model determines a unique proposition for the extensional sentences $\varphi$ and $\psi$ of $\L$ by extending $\vert{\cdot}$ with the following semantic clauses: 
    \begin{itemize}
      % \item[($p$)] $\vert{p}=\Vert{p}$ if $p\in\Lit_n$.
      \item[($\neg$)] $\vert{\tuple{\neg,\varphi}}=\neg\vert{\varphi}$.
      \item[($\wedge$)] $\vert{\tuple{\wedge,\varphi,\psi}}=\vert{\varphi}\wedge\vert{\psi}$.
      \item[($\vee$)] $\vert{\tuple{\vee,\varphi,\psi}}=\vert{\varphi}\vee\vert{\psi}$.
    \end{itemize}
    % The clauses above assign all extensional sentences to propositions.
  \item[\it Compatible Parts:] Given a world state $w$ together with any other state $s$, we may consider the set of maximal parts of $w$ which are compatible with $s$:\\ 
    $(w)_s\coloneq \set{t\sqsubseteq w:t\circ s \wedge \forall r\sqsubseteq w((r\circ s \wedge t \sqsubseteq r) \rightarrow t = r)}$.
  \item[\it Minimal Changes] Given a proposition $\tuple{V,F}$, we may define the set of all world states that result from minimally changing $w$ to include a state $s\in V$:\\ 
      $\tuple{V,F}^w\coloneq \set{w'\in W:\exists s\in V\exists t\in(w)_s(s.t\sqsubseteq w')}$.
  \item[\it Bilateral Notation:] Given an extensional sentence $\varphi$ and $n$-model $\M=\tuple{S,P,\Vert{\cdot}}$, it will be convenient to adopt the notation $\vert{\varphi}=\tuple{\vert{\varphi}^+,\vert{\varphi}^-}$.
  \item[\it Counterfactual Semantics:] Letting $\L_n^+=\tuple{\Lit_n,\neg,\wedge,\vee,\boxright}$ be a counterfactual language of rank $n$, the extensional sentence $\varphi$ and sentences $A$, $B$, and $C$ may be evaluated for truth at an $n$-model $\M$ together with a world $w$ as follows:
    \begin{itemize}
      \item[] $\M, w \vDash \varphi$ \textit{iff} there is some $s \sqsubseteq w$ where $s \in \vert{\varphi}^+$.
      \item[] $\M, w \vDash \neg A$ \textit{iff} $\M, w \nvDash A$.
      \item[] $\M, w \vDash A \wedge B$ \textit{iff} $\M, w \vDash A$ and $\M, w \vDash B$.
      \item[] $\M, w \vDash A \vee B$ \textit{iff} $\M, w \vDash A$ or $\M, w \vDash B$.
      \item[] $\M, w \vDash \varphi\boxright C$ \textit{iff} $\M, w' \vDash C$ whenever $w'\in\vert{\varphi}^w$.
    \end{itemize}
  \item[\it Condition:] We may read `$\M, w \vDash A$' as `$A$ is true at $w$ in $\M$'. 
    The aim will be to ask Z3 if there is an $n$-model $\M$ of $\L_n^+$ and world $w$ over its corresponding $n$-frame where $\M, w \vDash A$.
  % \item[\it Example:] $\M,w\vDash \tuple{\boxright, \tuple{\vee, \tuple{1}, \tuple{2}}, \tuple{3}}$ just in case $\M, w' \vDash \tuple{3}$ for every $w'\in\vert{\tuple{\vee, \tuple{1}, \tuple{2}}}^w$ where $\vert{\tuple{\vee, \tuple{1}, \tuple{2}}}^w=(\vert{\tuple{1}}\vee\vert{\tuple{2}})^w$, etc.
  \item[\it Satisfaction:] A world-model pair \textit{satisfies} a set of sentences $\Gamma$ just in case every sentence in $\Gamma$ is true in that world on that model.
  \item[\it Unsatisfiable:] A set of sentences $\Gamma$ is $n$-\textit{satisfiable} just in case there is a $n$-model $\M$ and world $w$ that satisfies $\Gamma$, and $n$-\textit{unsatisfiable} otherwise.
    % A world-model pair satisfies a sentence just in case it satisfies its singleton.
  \item[\it $n$-Entailment:] $\Gamma \vDash^n \varphi$ \textit{iff} $\Gamma\cup\set{\neg\varphi}$ is $n$-unsatisfiable. 
  % \item[\it Reduction:] Given a set of sentences $\Gamma$, we may consider the set of sentence letters $\Gamma_0$ that the sentences in $\Gamma$ contain, calling this $\Gamma$'s reduction. 
  % \item[\it Restriction:] Given an $n$-frame, we may restrict consideration to the models which assign the sentence letters in $\Gamma_0$ to propositions defined over the frame. This will be a comparatively small set
\end{enumerate}

% \noindent
% When $n$ is small, the number of states, $n$-frames, propositions, and world-states will be small.
% When $r$ is small, the space of models will be small.




\section{Logic Search}

This section aims to provide a systematic way to find $n$-entailments without attempting to conduct an exhaustive search over all $n$-entailments given the size of this space.



% This part aims to construct an efficient method for evaluating individual sets of sentences for unsatisfiability that are specifically provided, as well as constructing a search procedure for evaluating all sets of sentences for unsatisfiability, appropriately recording these findings.
% This section aims to deploy Z3 to check sets of sentences for $n$-unsatisfiability which may be used to define $n$-entailment.

% \begin{enumerate}
%   \item[\it Complexity:] The complexity of a sentence is the number of its operators, where this may be defined recursively. The complexity of a set of sentences is the sum of the complexity of its members.
%   \item[\it Checker:] Given a set of $r$ sentences $\Gamma$ and model order $n$, use Z3 to evaluate whether $\Gamma$ is $n$-satisfiable, storing all models that satisfy $\Gamma$ by updating a global file \texttt{rank-r} to record findings.
%     \begin{itemize}
%       \item Let \texttt{rank-r} be a file which begins empty and is updated to include entries for the sets of $r$ sentences that have been checked so far where entries are ordered according to their complexity.
%       \item Entries in \texttt{rank-r} should include: (1) a set of $r$ sentences $\Gamma$; (2) a model order number $n$ with a default value of 0; and (3) a table of countermodels organised by order if any.
%       \item Before checking whether a set of $r$ sentences $\Gamma$ is $n$-satisfiable, check to see if an entry exists in \texttt{rank-r}. If there is an entry for $\Gamma$ with order number $m\geq n$, then return `unsatisfiable' and halt. If there is an entry for $\Gamma$ with order number $m<n$, then check for satisfiable models of order $k$ for $m\leq k\leq n$. If there is no entry for $\Gamma$, then proceed to check whether $\Gamma$ is $n$-unsatisfiable. 
%     \end{itemize}
%   \item[\it Satisfiable:] If $\M$ is an $n$-model that has been found to satisfy the set of $r$ sentences $\Gamma$, store $\M$ in a new entry in \texttt{rank-r} if an entry for $\Gamma$ does not already exist, and otherwise add $\M$ to the entry in \texttt{rank-r} for $\Gamma$.
%   \item[\it Unsatisfiable:] If $\Gamma$ is found to be $n$-unsatisfiable, create an entry for $\Gamma$ in \texttt{rank-r} if none exists and store $n$ alongside $\Gamma$. If there is an entry for $\Gamma$ in \texttt{rank-r}, replace the stored model order number $m$ with $n$ if $n>m$.
%   % % \item[\it Stages:] The checker should proceed in stages, checking all models over the $1$-frames before proceeding to the $2$-frames and so on up to some specified limit $n$.
%   % \item[\it Exhaustive:] Search the space of all $k$-models for $k\leq n$ to see which satisfies $\Gamma$, storing those models in an output file which includes $\Gamma$. 
%   % \item[\it Incremental:] Search the space of all $k$-models for $k\leq n$ to see which satisfies $\Gamma$ first, storing that model in an output file which includes $\Gamma$ and prompting the user whether to halt, continue searching for the next model that satisfies $\Gamma$ and updating the output file, or else to conduct an exhaustive search of the space of $k$-models that satisfy $\Gamma$ for $k\leq n$.
%   % \item[\it Search:] For rank $r$, complexity $c$, multiplicity $m$, and order $n$, we may seek to determine which sets of $m$ sentences of rank $r$ and complexity $c$ are unsatisfiable over all $k$-frames for $k\leq n$. If a $(r,c,m)$-set of sentences $\Gamma$ is not satisfied by a model, store the hash for that model in the falsifier list for $\Gamma$ and continue searching for a model that satisfies $\Gamma$. If a model is found to satisfy $\Gamma$, store the hash for the model in question in the verifier list for $\Gamma$ and move to check the next set of sentences. If no model is found to satisfy $\Gamma$, store the hash for $\Gamma$ in a list of unsatisfiable sets of sentences and move to check the next set of sentences. 
%   % \item[\it Logic:] The unsatisfiable sets of sentences determine the $n$-entailments which may then be fed into a theorem prover. If no proof is found within time $t$, we may check the $n$-entailment for validity by hand. If valid, the $n$-entailment may be added to the basic rules for the logic.
% \end{enumerate}



% \section{Ordered Sentences}

% \begin{enumerate}
%   \item[\it Sentence Letters:] Let $\L_i=\set{p_j:j\in\N \text{ and } j\leq i}$ be the set of the first $i$ sentence letters. 
%   \item[\it Operators:] Let $\O=\set{\neg,\wedge,\vee,\boxright}$ be the set of primitive operators. 
%   \item[\it Sentences:] The well-formed sentences of a language with counterfactual and extensional operators do not permit counterfactual operators to occur in the antecedent of a counterfactual sentence. In order to devise versatile conventions it will be important to look at the syntax used in Prover9 or similar theorem provers.
%   \item[\it Rank:] A sentence has rank $r$ just in case it only includes sentence letters in $\L_r$. We may begin by restricting consideration to sentences with low rank, e.g., $r\leq5$ since rank will amplify the number of models.
%   \item[\it Complexity:] Second to rank, sentences can be ordered according to how many sentential operators they include, referring to this number as their complexity $c$. Rank should take president over complexity, though it is also important to restrict consideration to low complexity, e.g., $c\leq5$.
%   \item[\it Storage:] Beginning with rank $r=1$, sentences will need to be generated and stored in the order of their complexity before moving to rank $r=2$.
%   \item[\it Hash:] Each sentence should have a hash.
%   \item[\it $r$-Lists:] Beginning with $r=1$, the hash for each sentence should be stored in a list for $r=1$, ordered by complexity. We will need a separate $r$-list for each value $r\leq 5$ so that we may easily extend these lists to include higher complexity sentences. New lists for $r>5$ may be added later. 
%   \item[\it Bypass:] Each hash in an $r$-list should be stored with a bypass parameter that can be turned on or off so that sentences that are later deemed to be uninteresting can be skipped without removing them from the list.
%   \item[\it $r$-Sets:] Given a rank $r$, complexity $c$, and multiplicity $m$, populate and store all sets of at most $m$ hash codes for sentences with rank $k\leq r$ and complexity $l\leq c$ in a $(r,c,m)$-list which can be incrementally extended. 
%   \item[\it Valuations:] Each set of hash codes $\Gamma$ should be stored alongside a verifier list and a falsifier list that can later be populated with the hash codes for models which satisfy $\Gamma$ or which do not satisfy $\Gamma$, respectively. 
%     % \begin{itemize}
%     %   \item[\it Base:] Let $B_{(0,0,0)}=\set{\varnothing}$ be the set of all sets of $m=0$ sentences of rank $r=0$ and complexity $c=0$. 
%     %   \item[\it Rank:] Given $B_{(r,c,m)}$, let $B_{(r+1,c,m)}$ be the set of all sets of $m$ sentences of rank $r+1$ and complexity $c$ that do not occur in $B_{(r,c,m)}$. 
%     %   \item[\it Complexity:] Given $B_{(r,c,m)}$, let $B_{(r,c+1,m)}$ be the set of all sets of $m$ sentences of rank $r$ and complexity $c+1$ that do not occur in $B_{(r,c,m)}$. 
%     %   \item[\it Multiplicity:] Given $B_{(r,c,m)}$, let $B_{(r,c,m+1)}$ be the set of all sets of $m+1$ sentences of rank $r$ and complexity $c$ that do not occur in $B_{(r,c,m)}$.
%     % \end{itemize}
%   % \item[\it Aggregate:] In running the model checker on all sentences, we may begin with the list for $r=1$ moving up in complexity before moving to $r=2$, etc.
%   % \item[\it Copy:] If a set of sentences can be shown to be satisfiable, we will later want to store the hash for that sentence along with the hash for the model in a new list of falsifiable sentences of rank $r$. Something similar may be done if a sentence with rank $r$ is true in a model, storing the ordered pair of hashes in a list of satisfiable sentences of rank $r$. 
%   % \item[\it Summary:] We will need a separate list of sentences for at least each pair $\tuple{r,c}$, as well as a list of hash codes for each  ordered by complexity of the corresponding sentence. Once we start evaluating sentences at models, new lists will need to be created to record the findings.
% \end{enumerate}





\section{Representations}

Organize and print found $n$-entailments as well as graphically representing countermodels.

  % \begin{enumerate}
  % \item If a sentence is false at a model, the sentence should be moved to a list of invalid sentences where the hash of the associated model is included 
  % \end{enumerate}



% \section{Representations}



% \vfill
%
% \bibliographystyle{Phil_Review} %%bib style found in bst folder, in bibtex folder, in texmf folder.
% \nobibliography{Zotero} %%bib database found in bib folder, in bibtex folder


\end{document}
